\documentclass{article}

\usepackage{amsthm}
\usepackage{amsmath}
\usepackage{amsfonts}
\usepackage{amssymb}
\usepackage{amscd}
\usepackage{mathrsfs}
\usepackage{bm}
\usepackage{url,hyperref}

\newtheorem{thm}{Theorem}[section]
\newtheorem{cor}[thm]{Corollary}
\newtheorem{lem}[thm]{Lemma}
\newtheorem{prop}[thm]{Proposition}
\newtheorem{conj}[thm]{Conjecture}
\newtheorem{assumption}[thm]{Assumption}
\theoremstyle{definition}
\newtheorem{defn}[thm]{Definition}%[section]
\newtheorem{example}[thm]{Example}
\newtheorem{remark}[thm]{Remark}
\newtheorem{algorithm}[thm]{Algorithm}

\newcommand{\R}{\mathbb{R}}
\newcommand{\C}{\mathbb{C}}
\newcommand{\Z}{\mathbb{Z}}
\newcommand{\N}{\mathbb{N}}
\newcommand{\Q}{\mathbb{Q}}
\newcommand{\A}{\mathbb{A}}
\newcommand{\PP}{\mathbb{P}}
\newcommand{\Prob}{{\rm Prob}}
\newcommand{\pa}{{\rm pa}}
\newcommand{\nd}{{\rm nd}}
\def\ci{\perp\!\!\!\perp}

\title{A computational algebraic geometry package for graphical models}
\author{Luis David Garcia-Puente, \\Sonja Petrovi\'c, Seth Sullivant}
%\date{}
\date{\today}

\begin{document}
\maketitle
\begin{abstract}
The abstract is not to exceed 150 words.

Good example of a recent paper:\\ \url{http://j-sag.org/Volume4/jsag-1-2012.pdf}
\end{abstract}




\section{Graphical models}
{\bf to put here: }some math background and including an example of some basic functionality.
Motivation: why do we have this package?
\begin{itemize}
\item background 
\item CI defn
\item parametric representation
\item discrete and gaussian r.v.'s
\end{itemize}

%\small{
\begin{verbatim}
Macaulay2, version 1.4
i1 : G = digraph {{a,{b,c}}, {b,{c,d}}, {c,{}}, {d,{}}}
o1 = Digraph{a => set {b, c}}
             b => set {c, d}
             c => set {}
             d => set {}
\end{verbatim}
%}

\section{Computing conditional independence ideals}

\begin{itemize}
\item What do they mean, in general? 
\item  families of CI statements associated to graphs
\item trek separation
\end{itemize}


\section{Computing the vanishing ideal of a model}

\begin{itemize}
\item from parametric representation
\item how does it compare to CI ideal
\item primary decomp.
\end{itemize}




\section{Technical discussion (?) }
This package requires Graphs.m2, as a consequence it can do
computations with graphs whose vertices are not necessarily labeled by
integers. 


\section*{Acknowledgements} {\bf TO DO: -- other authors of the package!....} -- Dan Grayson and Amelia Taylor.




\begin{thebibliography}{20}

\bibitem{theRpackage!}
Gabriel C. G. de Abreu, Rodrigo Labouriau, David Edwards, High-dimensional Graphical Model Search with gRapHD R Package, \href{http://arxiv.org/abs/0909.1234}{arXiv:0909.1234v4}
{\bf CITE ME!!!}

\bibitem{DSS} 
M.~Drton, B.~Sturmfels and S.~Sullivant, (2009) \emph{Lectures on Algebraic Statistics}, Birkhauser 

\bibitem{MortonSturm}
I. Onur Filiz, Xin Guo, Jason Morton, Bernd Sturmfels, Graphical models for correlated defaults, \href{http://arxiv.org/abs/0809.1393}{arXiv:0809.1393v1}
{\bf CITE ME!!!}

\bibitem{GSS} L.~D.~Garcia, M.~Stillman and B.~Sturmfels: Algebraic geometry of Bayesian networks, {\em J. Symbolic Comput.}
  {\bf 39} (2005) 331--355.
  
\bibitem{GeigerMeekSturm}
Dan Geiger, Christopher Meek, Bernd Sturmfels, On the toric algebra of graphical models, \href{http://arxiv.org/abs/math/0608054}{arXiv:math/0608054v1}
{\bf CITE ME!!!}

\bibitem{HaraTakemura}
Hisayuki Hara, Satoshi Aoki, Akimichi Takemura, Minimal and minimal invariant Markov bases of decomposable models for contingency tables, Bernoulli 2010, Vol. 16, No. 1, 208-233  (preprint FOR OUR REFERENCE: \href{http://arxiv.org/abs/math/0701429}{arXiv:math/0701429v3})
{\bf CITE ME?}

\bibitem{Lauritzen}
S.~Lauritzen,  (1996) \emph{Graphical models}, Oxford University Press

\bibitem{Elena}
E.~Stanghellini, B.~Vantaggi, On the identification of discrete graphical models with hidden nodes, \href{http://arxiv.org/abs/1009.4279}{	arXiv:1009.4279v1}
{\bf DO WE NEED THIS REFERENCE?}

\bibitem{S} S.~Sullivant: Algebraic geometry of Gaussian Bayesian
  networks, {\em Adv. in Appl. Math.} {\bf 40} (2008) 482--513.

\bibitem{STD}
S.~Sullivant, K.~Talaska, and J.~Draisma: Trek separation for Gaussian
graphical models, {\em Ann. Statist.} {\bf 38} (2010) 1665--1685. 

\bibitem{GPSS} L.~D.~Garcia-Puente, S.~Spielvogel and S.~Sullivant: Identifying
causal effects with computer algebra, to appear in {\em Proceedings of the 26th
Conference of Uncertainty in Artificial Intelligence}.

\end{thebibliography}
\end{document}


