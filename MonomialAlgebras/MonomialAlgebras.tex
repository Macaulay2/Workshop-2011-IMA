\documentclass[11pt,a4paper]{amsart}%
\usepackage{hyperref}
\usepackage{amsfonts}
\usepackage{amsmath}
\usepackage{amssymb}
\usepackage{graphicx}
\usepackage{algorithm}
\usepackage{a4wide}
\usepackage{algorithmic}%
\setcounter{MaxMatrixCols}{30}
%TCIDATA{OutputFilter=latex2.dll}
%TCIDATA{Version=5.00.0.2606}
%TCIDATA{LastRevised=Thursday, June 02, 2011 11:54:27}
%TCIDATA{<META NAME="GraphicsSave" CONTENT="32">}
%TCIDATA{<META NAME="SaveForMode" CONTENT="1">}
%TCIDATA{BibliographyScheme=Manual}
\newtheorem{theorem}{Theorem}[section]
\newtheorem*{Theorem}{Theorem}
\newtheorem{lemma}[theorem]{Lemma}
\newtheorem{proposition}[theorem]{Proposition}
\newtheorem{corollary}[theorem]{Corollary}
\theoremstyle{definition}
\newtheorem{definition}[theorem]{Definition}
\newtheorem{example}[theorem]{Example}
\theoremstyle{remark}
\newtheorem{remark}[theorem]{Remark}
\numberwithin{equation}{section}
\theoremstyle{plain}
\newtheorem{acknowledgement}{Acknowledgement}
\newtheorem{axiom}[theorem]{Axiom}
\newtheorem{case}[theorem]{Case}
\newtheorem{claim}{Claim}
\newtheorem{conclusion}{Conclusion}
\newtheorem{condition}{Condition}
\newtheorem{conjecture}[theorem]{Conjecture}
\newtheorem{criterion}{Criterion}
\newtheorem{exercise}{Exercise}
\newtheorem{notation}{Notation}
\newtheorem{problem}{Problem}
\newtheorem{solution}{Solution}
\newtheorem{summary}{Summary}
\newcommand{\erz}[1]{\langle#1\rangle}
\newcommand{\de}{{\rm deg}\,}
\newcommand{\maxk}[1]{\left\{#1\right\}}
\renewcommand{\algorithmicrequire}{\textbf{Input:}}
\renewcommand{\algorithmicensure}{\textbf{Output:}}
\begin{document}
\title[Decomposition of semigroup rings]{Decomposition of semigroup rings}
\author{Janko B\"{o}hm}
\address{Department of Mathematics, Universit\"{a}t des Saarlandes, Campus E2 4 \\
D-66123 Saarbr\"{u}cken, Germany}
\email{boehm@math.uni-sb.de}
\author{David Eisenbud}
\address{Dept of Mathematics, University of California, Berkeley, CA 94720, USA}
\email{de@msri.org}
\author{Max J. Nitsche}
\address{Max-Planck-Institute for Mathematics in the Sciences, Inselstrasse 22, 04103
Leipzig, Germany}
\email{nitsche@mis.mpg.de}
\thanks{}
\date{\today}
\keywords{Semigroup rings, Castelnuovo-Mumford regularity, Eisenbud-Goto conjecture, Computational commutative algebra.}
\subjclass[2010]{Primary 13D45; Secondary 13P99, 13H10.}



\begin{abstract}

We show how a semigroup ring $R[B]$ can be decomposed into a direct sum of submodules in $R[G(A)]$ over $R[A]$, where $A\subseteq B$ are cancellative abelian semigroups and $R$ is an integral domain. In the case of a finite extension of positive affine semigroup rings we obtain an algorithm computing the decomposition. This can be applied to characterize ring theoretic properties in the simplicial case. Moreover, the regularity of homogeneous affine semigroup rings can be computed in terms of the decomposition. This leads to a new fast algorithm, which we use to confirm the Eisenbud-Goto conjecture for some cases. All algorithms are implemented in the \textsc{Macaulay2} package \textsc{MonomialAlgebras}.

\end{abstract}
\maketitle


\section{Introduction}


Let $R$ be an integral domain and $(B,+)$ a cancellative abelian semigroup. Denote by $G(B)$ the group generated by $B$, and by $R[B]$ the semigroup ring associated to $B$, that is the free $R$-module with basis consisting of the symbols $t^a$ for $a\in B$, and multiplication given by the $R$-bilinear extension of $t^a\cdot t^b=t^{a+b}$. Hoa and St\"uckrad presented in \cite{HSCM} a decomposition of homogeneous simplicial affine semigroup rings into a direct sum of certain monomial ideals. They used this to bound the Castelnuovo-Mumford regularity of the semigroup ring. We will generalize this result in Theorem~\ref{decomposition} showing that for a subsemigroup $A\subseteq B$ the $R[A]$-module $R[B]$ is isomorphic to a direct sum of submodules $I_{g}\subseteq R[G(A)]$ indexed by elements $g\in G:= G(B)/G(A)$. Here we consider the $R[A]$-module structure on $R[B]$ given by inclusion.\\

We then focus on the case that $K$ is a field and $A\subseteq B\subseteq \mathbb N^m$ are affine semigroups such that $K[B]$ is a finite $K[A]$-module. In this case the number of submodules $I_g$ is finite, moreover, we can always choose them to be in $K[A]$. This allows us to give an algorithm computing the decomposition, which is implemented in our \textsc{Macaulay2} \cite{M2} package \textsc{MonomialAlgebras} \cite{BEN}. In case that $B$ is simplicial, that is, $B$ generates a simplicial cone, many ring theoretic properties of $K[B]$ can be described in terms of the semigroup $B$, such as being Cohen-Macaulay, Buchsbaum, Gorenstein, normal, or seminormal, see \cite{GSRBB, MHCM, HRLC, PHDPL, RSHFGA}. If $A$ is chosen to be generated by elements on the extremal rays of $B$, all these properties can be characterized in terms of the decomposition, see Proposition~\ref{char}. Using this we can provide functions to test those properties efficiently.\\ 

In Section~\ref{sec simplicial homogeneous case} we consider the case that $B$ is homogeneous, that is, there is a grading on $K[B]$ in which every element in the minimal generating set ${\rm Hilb}(B)$ of $B$ has degree $1$. The Castelnuovo-Mumford regularity of $K[B]$ (see Section~\ref{sec simplicial homogeneous case}) is usually computed from a minimal graded free resolution of $K[B]$ as a canonical $K[x_1,\ldots,x_n]$-module, where $n$ is the cardinality of ${\rm Hilb}(B)$. The main problem is that this free resolution could have length $n$, and if $n$ is large this computation becomes nearly impossible, say, for $n\geq15$. By using the decomposition of $K[B]$ we can compute its regularity in terms of the regularity of the (monomial) ideals $I_g$ in $K[A]$, under the assumption that $A$ is generated by degree one elements. Thus, the problem of computing the regularity of $K[B]$ comes down to computing a minimal graded free resolution of $I_g$ which could have length $d$, where $d$ is the cardinality of ${\rm Hilb}(A)$. Since the decomposition can be obtained very efficiently by our package also in high codimension, our regularity computation is typically much faster than the usual one, see Section~\ref{sec simplicial homogeneous case} for timings. This enables us to test the Eisenbud-Goto conjecture \cite{EG} for certain affine semigroup rings in Theorem~\ref{testegbis5}.




\section{Decomposition}


For a subset $X$ in $G(B)$ we denote $t^X:=\{t^x\mid x\in X\}$.


\begin{theorem}\label{decomposition} 

Let $A\subseteq B$ be cancellative abelian semigroups, and let $R$ be an integral domain. The $R[A]$-module $R[B]$ is isomorphic to the direct sum of submodules $I_{g}\subseteq R[G(A)]$ indexed by elements $g\in G:= G(B)/G(A)$.

\end{theorem}
\begin{proof}
Let
\[
\Gamma_{g}':=\{b\in B\mid b\equiv g\text{ mod }G(A)\}.
\]
By construction, we have
\[
R[B]=\bigoplus_{g\in G} R\cdot t^{\Gamma_{g}'}.
\]
For each $g\in G$, choose an element $h_{g}\in G(B)$ such that $h_g\equiv g\text{ mod }G(A)$. The module $R\cdot t^{\Gamma_{g}'}$ is an $R[A]$-submodule of $R[B]$ and, as such, it is isomorphic to
\[
I_{g}:=R\cdot\{t^{b-h_{g}}\mid b\in\Gamma_{g}'\}\subseteq R[G(A)]\text{.}
\]
\end{proof}


Note also that the collection of indexed ideals $\{I_{g}\subseteq
R[G(A)]\}_{g\in G}$, together with the elements $h_{g}\in G(B)$ are sufficient
to determine the ring structure of $R[B]$; indeed, if $x\in I_{g_{1}}$ and
$y\in I_{g_{2}}$ and $xy=z$ as elements of $R[G(A)]$ then as elements in the
decomposition of $R[B]$%
\[
x\cdot_{R[B]}y=\frac{t^{h_{g_{1}}}t^{h_{g_{2}}}}{t^{h_{g_{1}+g_{2}}}}z\in I_{g_{1}+g_{2}}.
\]
In the following we consider the case that the coefficient ring is a field, denoted by $K$, and $A\subseteq B\subseteq \mathbb N^m$ are affine semigroups, that is, $A$ and $B$ are finitely generated submonoids of $(\mathbb N^m,+)$. We can always choose a minimal subset $B_A$ of $B$ such that $K[B]=K[A]\cdot t^{B_A}$ as $K[A]$-modules. Note that
$$
B_A=\{x\in B\mid x-a \notin B~\forall a\in A\setminus\{0\}\}.
$$
Let $g\in G$ and $\Gamma_g=\{b\in B_A \mid b\equiv g\text{ mod }G(A)\}$. By construction $t^{\Gamma_g}$ is a minimal generating set of $K\cdot t^{\Gamma_{g}'}$ as a $K[A]$-module. Thus, to compute the decomposition we want the set $B_A$ to be finite. This is clearly equivalent to $K[B]$ being a finite $K[A]$-module. Moreover, it is equivalent to $C(A)=C(B)$, where $C(X)$ denotes the cone spanned by $X$ in $\mathbb Q^m$. Note that if $B_A$ is finite, then $G(B)/G(A)$ is also finite.\\ 

The second claim follows from the observation that in case $C(A)\subsetneqq C(B)$ we can choose an element $x\in B$ on a ray of $C(B)$ not in $C(A)$. So $nx\in B_A$ for all $n\in\mathbb N^+$, hence $B_A$ is not finite. Conversely, if $C(A)=C(B)$ we get that $B_A$ is finite, since for all $b\in B$ there is an $n\in\mathbb N^+$ such that $nb\in A$.\\ 

From these observations we obtain Algorithm~\ref{algo1} computing the set $B_A$ and the decomposition.



\begin{algorithm}[H]           
\caption{Decompose monomial algebra}  
\begin{algorithmic}[1]\label{algo1}

\REQUIRE  A homogeneous monomorphism
\[
\psi:K[y_{1},\ldots,y_{d}] \rightarrow K[x_{1},\ldots,x_{n}]
\]
of $\mathbb N^m$-graded polynomial rings with $\deg x_{i}=b_{i}$ and $\deg y_{j}=e_{j}$ specifying the affine semigroups $A=\langle e_{1},\ldots,e_{d}\rangle\subseteq B=\left\langle b_{1},\ldots,b_{n}\right\rangle\subseteq \mathbb N^m$ with $C(A)=C(B)$.
\ENSURE Ideals $I_{g}\subseteq K[A]$ and shifts $h_g\in G(B)$ with
\[
K[B] \cong\bigoplus\nolimits_{g\in G} I_{g} (-h_{g})
\]
as $\mathbb Z^m$-graded $K[A]$-modules, where $G=G(B)/G(A)$.
\STATE Compute the toric ideal $I_B={\rm ker}\varphi$ associated to $B$, where
\[
\varphi: K[x_{1},\ldots,x_{n}] \rightarrow K[B], \quad x_i\mapsto t^{b_i}.
\]
\STATE Compute a monomial $K$-basis $v_{1},\ldots,v_{r}$ of
\[
K[x_{1},\ldots,x_{n}]/(I_B+\psi(\langle y_{1},\ldots,y_{d}\rangle)).
\]
\STATE Reduce $\de v_{i}$ modulo $G(A)$ to obtain $g_{1},\ldots,g_{f}\in G(B)$ representing the elements of $G$.
\STATE We have
\[
B_A= \{\de v_{1},\ldots,\de v_{r}\}=\bigcup\nolimits_{i=1}^f\Gamma_{\bar g_i},
\]
where $\Gamma_{\bar g_i}$ are the equivalence classes on $B_A$ modulo $G(A)$.
\FOR{$i=1$ to $f$}
\FOR{$v\in \Gamma_{\bar g_i}$}
\STATE Fix some elements $c_{vj}\in\mathbb Z$ such that
\[
v-g_i =\sum\nolimits_{j=1}^d c_{vj}e_j.
\]
\ENDFOR
\STATE Let $h_{\bar g_i}=g_i+\sum\nolimits_{j=1}^d \min\maxk{c_{vj}\mid v\in \Gamma_{\bar g_i}}e_j$ and $I_{\bar g_i}=\left\langle t^{v -h_{\bar g_i}}\mid v\in \Gamma_{\bar g_i}
\right\rangle $.
\ENDFOR
\RETURN $I_{\bar g_i}$ and $h_{\bar g_i}$ for $i=1,\ldots,f$.

\end{algorithmic}
\end{algorithm}

Note that for $x\in\Gamma_{\bar{g}_{i}}$ we always have
\[
x-h_{\bar{g}_{i}}=\sum\nolimits_{j=1}^{d}\left(  c_{xj}-\min\left\{
c_{vj}\mid v\in\Gamma_{\bar{g}_{i}}\right\}  \right)  e_{j}=\sum
\nolimits_{j=1}^{d}n_{j}e_{j}\in A,
\]
for some $n_{j}\in\mathbb{N}$, hence $I_{\bar g_i}$ are monomial ideals in $K[A]$.


\begin{example}\label{zerlbeispiel}

Consider $B=\erz{(2, 0, 3), (4, 0, 1), (0, 2, 3), (1, 3, 1), (1, 2, 2)}\subset\mathbb N^3$ and the submonoid $A=\erz{(2, 0, 3), (4, 0, 1), (0, 2, 3), (1, 3, 1)}$. We get the decomposition of $B_A$ into equivalence classes
$B_{A}=\{0,(2,4,4)\}\cup\{(1,2,2),(3,6,6)\}$. Choosing $h_1=(-4,-2,-4)$ and $h_2=(-3,0,-2)$ in $G(B)$ we have
\begin{align*}
K[B] &\cong K[A]\{t^{(4,2,4)}, t^{(6, 6, 8)}\} (-h_1) \oplus K[A]\{t^{(4, 2, 4)}, t^{(6, 6, 8)}\} (-h_2)\\
&\cong \erz{xy,w^2z^2} (-h_1) \oplus \erz{xy,w^2z^2} (-h_2),
\end{align*}
where $K[A]\cong K[w,x,y,z]/\erz{w^3z^2-x^2y^3}$. 



\end{example}


\begin{example}
Using our implementation of Algorithm $1$ in the \textsc{Macaulay2} package \textsc{MonomialAlgebras} we compute the decomposition of $K[B]$ over $K[A]$ for the following example:

\texttt{i1: loadPackage "MonomialAlgebras";}

\texttt{i2: A = \{\{5,0\},\{0,5\}\};}

\texttt{i3: B = \{\{5,0\},\{0,5\},\{1,4\},\{4,1\}\};}

\texttt{i4: S = QQ[x\_0 .. x\_3, Degrees=>B];}

\texttt{i5: P = QQ[x\_0, x\_1, Degrees=>A];}

\texttt{i6: f = map(S,P);}

\texttt{i7: decomposeMonomialAlgebra f}

\texttt{
\begin{tabular}
[c]{llll}%
\hspace{-0.15in}\texttt{o7: HashTable\{} & \{0,0\} & \texttt{=> \{ ideal 1}, &
\{0,0\} \}\\
& \{1,-1\} & \texttt{=> \{ }ideal 1, & \{1,4\} \}\\
& \{-1,1\} & \texttt{=> \{ }ideal 1, & \{4,1\} \}\\
& \{2,-2\} & \texttt{=> \{ }ideal ( x$_{0},$ x$_{1}^{2}$ ), & \{2,3\} \}\\
& \{-2,2\} & \texttt{=> \{ }ideal ( x$_{0}^{2},$ x$_{1}$ ), & \{3,2\} \}
\end{tabular}
}\\

The keys of the hash table represent the elements of $G$.
\end{example}




\section{Ring theoretic properties}\label{sec simplicial case}



Recall that an affine semigroup $B\subseteq\mathbb N^m$ is called simplicial if it spans a simplicial cone, or equivalently, there are linearly independent elements $e_1,\ldots,e_d\in B$ with $C(B)=C(\{e_1,\ldots,e_d\})$.



\begin{proposition}\label{char}

Let $K$ be a field, $B\subseteq \mathbb N^m$ a simplicial affine semigroup, and let $A$ be the submonoid of $B$ which is generated by linearly independent elements $e_1,\ldots,e_d$ of $B$ with \mbox{$C(A)=C(B)$}. Let $B_A$ be as above, and $K[B]\cong\bigoplus_{g\in G} I_{g}(-h_g)$ be the output of Algorithm~\ref{algo1} with respect to $A\subseteq B$. We have:

\begin{enumerate}

\item The depth of $K[B]$ is the minimum of the depths of the ideals $I_g$.

\item $K[B]$ is Cohen-Macaulay if and only if every ideal $I_g$ is equal to $K[A]$.

\item $K[B]$ is Gorenstein if and only if $K[B]$ is Cohen-Macaulay and the set of shifts $\{h_g\}_{g\in G}$ has exactly one maximal element with respect to $\leq$ given by $x\leq y$ if there is an element $z\in B$ such that $x+z=y$.

\item $K[B]$ is Buchsbaum if and only if every proper ideal $I_g$ is equal to the homogeneous maximal ideal of $K[A]$ and $h_g+b\in B$ for all $b\in B\setminus\{0\}$.

\item $K[B]$ is normal if and only if for every element $x$ in $B_{A}$ there exist $\lambda_{1},\ldots,\lambda_{d}\in\mathbb{Q}$ with $0\leq\lambda_{i}<1$ such that $x=\sum_{i=1}^{d}\lambda_{i}e_{i}$.

\item $K[B]$ is seminormal if and only if for every element $x$ in $B_{A}$ there exist $\lambda_{1},\ldots,\lambda_{d}\in\mathbb{Q}$ with $0\leq\lambda_{i}\leq1$ such that $x=\sum_{i=1}^{d}\lambda_{i}e_{i}$.

\end{enumerate}

\end{proposition}
\begin{proof}


For every $x\in G(B)$ there are uniquely determined elements $\lambda_{1}^{x},\ldots,\lambda_{d}^{x}
\in\mathbb{Q}$ such that $x=\sum_{j=1}^{d}\lambda_{j}^{x}e_{j}$. Then by construction
\[
h_{g}=\sum\nolimits_{j=1}^{d}\min\left\{ \lambda^{v}_{j}\mid v\in\Gamma_{g}\right\} e_{j}.
\]
Assertion (1) and (2) follow immediately; (2) was already proved in \cite[Theorem\,6.4]{RSHFGA}. Assertion (3) can be found in \cite[Corollary\,6.5]{RSHFGA}.\smallskip\newline
(4) Let $I_g$ be a proper ideal, equivalently, $\#\Gamma_g\geq2$. The ideal $I_g$ is equal to the homogeneous maximal ideal of $K[A]$ and $h_g+b\in B$ for all $b\in B\setminus\{0\}$ if and only if $\Gamma_g=\{m+e_{1},\ldots,m+e_{d}\}$ for some $m$ with $m+b\in B$ for all $b\in B\setminus\{0\}$ and this is equivalent to $K[B]$ is Buchsbaum, by \cite[Theorem\,9]{GSRBB}.\smallskip\newline
(5) We set $D_{A}=\{x\in G\left(  B\right)  \mid x=\sum_{i=1}^{d}\lambda_{i}%
e_{i},\lambda_{i}\in\mathbb{Q}$ and $0\leq\lambda_{i}<1~\forall i\}$. The ring $K[B]$ is normal if and only if $B=C(B)\cap G(B)$ by \cite[Proposition\,1]{MHCM}. We need to show that $C(B)\cap G(B)\subseteq B$ if and only if $B_A\subseteq D_A$. We have $B_{A}\subseteq D_{A}$ if and only if $D_{A}\subseteq B_{A}$, since $B_{A}$ has $\#G=\#D_{A}$ equivalence classes and by definition of $B_{A}$. Note that $D_{A}\subseteq C(B)\cap G(B)$ and $D_{A}\cap B\subseteq B_{A}$. The assertion follows from the fact that every element $x\in C(B)\cap G(B)$ can be written as $x=x^{\prime}+\sum_{i=1}^{d}n_{i}e_{i}$ for some $x^{\prime}\in D_{A}$ and $n_{i}\in\mathbb{N}$.\smallskip\newline
(6) We set $\bar D_{A}:=\{x\in B \mid x =\sum_{i=1}^{d} \lambda_{i}e_{i}, \lambda_{i}\in\mathbb{Q }\mbox{ and }0\leq\lambda_{i}\leq1~\forall i\}$. By \cite[Proposition\,5.32]{HRLC} and \cite[Theorem\,4.1.1]{PHDPL} $K[B]$ is seminormal if and only if $B_A\subseteq\bar D_A$, provided that ${e_{1},\ldots,e_{d}}\in{\rm Hilb}(B)$. Otherwise there is a $k\in\{1,\ldots,d\}$ with $e_{k}=e_{k}^{\prime}+e_{k}^{\prime\prime}$ and $e_{k}^{\prime},e_{k}^{\prime\prime}\in B\setminus\{0\}$. We set $A^{\prime}=\langle e_{1},\ldots,e_{k}^{\prime},\ldots,e_{d}\rangle$ and $A^{\prime\prime}=\langle e_{1},\ldots,e_{k}^{\prime\prime},\ldots,e_{d}\rangle$. Clearly $C(A)=C(A^{\prime})=C(A^{\prime\prime})$. We need to show that $B_A\subseteq \bar D_A$ if and only if $B_{A'}\subseteq \bar D_{A'}$.  Let $x\in B_{A}\setminus\bar D_{A}$. If $x-e_{k}^{\prime}\notin B$, then $x\in B_{A^{\prime}}\setminus\bar D_{A^{\prime}}$. If $x-e_{k}^{\prime}\in B$, then $x-e_{k}^{\prime}\in B_{A^{\prime\prime}}\setminus\bar D_{A^{\prime\prime}}$. Let $x\in B_{A'}\setminus \bar D_{A'}$, say $x=\sum _{j\not=k}\lambda_je_j+\lambda_ke_k'$ and $\lambda_j>1$ for some $j$. If $j\not=k$, then $x\in B_A\setminus \bar D_A$. Let $j=k$; consider the element $y=x+e_{k}''-\sum_{j\not=k}n_je_j\in B$ for some $n_j\in \mathbb N$ such that $\sum_{j\not=k} n_j$ is maximal. It follows that $y\in B_A\setminus\bar D_A$ and we are done.
\end{proof}



\begin{example}[Macaulay-Curves]

Consider $B=\langle(\alpha,0), (\alpha-1,1), (1,\alpha-1), (0,\alpha)\rangle\subseteq \mathbb N^2$ and set $A=\langle(\alpha,0),(0,\alpha)\rangle$, say $K[A]=K[x,y]$. Note that we have $\alpha$ equivalence classes. We get
\[
K[B]\cong K[x,y]^{3}\oplus\langle x^{\alpha-3},y\rangle\oplus\langle
x^{\alpha-4},y^{2}\rangle\oplus\ldots\oplus\langle x,y^{\alpha-3}\rangle,
\]
as $K[x,y]$-modules, where the shifts are omitted. Hence each ideal of the form $\langle x^{i},y^{j}\rangle$, $1\leq i,j\leq\alpha-3$ with $i+j=\alpha-2$ appears exactly once in the decomposition. Hence $K[B]$ is not Buchsbaum for $\alpha>4$, since $\langle x^{\alpha-3},y\rangle$ is a direct summand. In case that $\alpha=4$ there is only one proper ideal $I_{4}=\langle x,y\rangle$ and $h_{4}=(2,2)$; in fact $(2,2)+B\setminus\{0\}\subseteq B$ and therefore $K[B]$ is Buchsbaum. It follows immediately that $K[B]$ is Cohen-Macaulay for $\alpha\leq3$, Gorenstein for $\alpha\leq2$, seminormal for $\alpha\leq3$, and normal for $\alpha\leq3$. Note that we could also decompose $K[B]$ over the subring $K[A]$, where $A=\langle(2\alpha,0),(0,2\alpha)\rangle=K[x,y]$, for $\alpha=4$ we would get
\[
K[B]\cong K[x,y]^{15} \oplus\langle x,y\rangle
\]
and the corresponding shift of $\langle x,y\rangle$ is again $(2,2)$.
\end{example}




\begin{example}[\cite{PHDPL}]

Let $B=\langle(1,0,0),(0,1,0),(0,0,2),(1,0,1),(0,1,1)\rangle\subset\mathbb N^3$, moreover, let \mbox{$A=\langle(1,0,0),(0,1,0),(0,0,2)\rangle$}, say $K[A]=K[x,y,z]$. We have
\[
K[B]\cong K[B]\oplus\langle x,y\rangle(-(0,0,1)),
\]
as $\mathbb{Z}^{3}$-graded $K[A]$-modules. Hence $K[B]$ is not Buchsbaum,
since $\langle x,y\rangle$ is not maximal; moreover, $K[B]$ is seminormal, but not normal.

\end{example}




\begin{example}
\label{ex gorenstein}
Consider the semigroup $B=\langle(1,0,0),(0,2,0),(0,0,2),(1,0,1),(0,1,1)\rangle\subset\mathbb N^3$, and set $A=\langle(1,0,0),(0,2,0),(0,0,2)\rangle$. We get
\[
K[B]\cong K[A]\oplus K[A](-(1,0,1))\oplus K[A](-(0,1,1))\oplus K[A](-(1,1,2)).
\]
Hence $K[B]$ is Gorenstein, since $(1,0,1)+(0,1,1)=(1,1,2)$. Moreover, $K[B]$ is not normal, since $(1,0,1)=(1,0,0)+\frac{1}{2}(0,0,2)$, but seminormal.

\end{example}




\begin{example}
We illustrate our implementation of the characterizations given in Proposition~\ref{char} at Example \ref{ex gorenstein}:

\texttt{i1: B = \{\{1,0,0\},\{0,2,0\},\{0,0,2\},\{1,0,1\},\{0,1,1\}\};}

\texttt{i2: isGorensteinMA B}

\texttt{o2: true}

\texttt{i3: isNormalMA B}

\texttt{o3: false}

\texttt{i4: isSeminormalMA B}

\texttt{o4: true}

\noindent Note that there are also commands \texttt{isCohenMacaulayMA} and \texttt{isBuchsbaumMA} available testing the Cohen-Macaulay and the Buchsbaum property, respectively.

\end{example}




\section{Regularity\label{sec simplicial homogeneous case}}

Let $K$ be a field and $R=K[x_{1},\ldots,x_{n}]$ be a standard graded
polynomial ring, that is, \mbox{$\de x_i=1$} for all $i=1,\ldots,n$. Moreover,
let $R_{+}$ be the homogeneous maximal ideal of $R$, and let $M$ be a finitely
generated graded $R$-module. We define the Castelnuovo-Mumford regularity
$\mathrm{reg}M$ of $M$ by
\[
\mathrm{reg}M:=\max\left\{  a(H_{R_{+}}^{i}(M))+i\mid i\geq0\right\}  ,
\]
where $a(H_{R_{+}}^{i}(M)):=\max\left\{  n\mid\lbrack H_{R_{+}}^{i}%
(M)]_{n}\not =0\right\}  $ and $a(0)=-\infty$; $H_{R_{+}}^{i}(M)$ denotes the
$i$-th local cohomology module of $M$ with respect to $R_{+}$. Note that
$\mathrm{reg}M$ can also be computed in terms of the shifts in a minimal
graded free resolution of $M$. An important application of the regularity is
that it can be used to bound the degrees in certain minimal Gr\"{o}bner bases
by \cite{BSCDM}. Thus, it is of interest to compute or bound the regularity of
a homogeneous ideal. The following conjecture (Eisenbud-Goto) was made in
\cite{EG}: If $K$ is algebraically closed and $I$ is a homogeneous prime ideal
of $R$ then for $S=R/I$%
\[
\mathrm{reg}S\leq\mathrm{deg}S-\mathrm{codim}S,
\]
where $\mathrm{deg}S$ denotes the degree of $S$ and $\mathrm{codim}%
S:=\mathrm{dim}_{K}\left[  S\right]  _{1}-\mathrm{dim}S$. This has been proved
for several cases including dimension $2$ by Gruson, Lazarsfeld, and Peskine
\cite{GLP}, the Buchsbaum case by St\"{u}ckrad and Vogel \cite{EGBB} (see also
\cite{EGCM}), for $\mathrm{deg}S\leq\mathrm{codim}S+2$ by Hoa, St\"{u}ckrad,
and Vogel \cite{HSVC2}, and for smooth surfaces in characteristic zero by
Lazarsfeld \cite{LSMSD3}. There is also a stronger version in which $S$ is
only required to be reduced and connected in codimension $1$; this version has
been proved for curves by Giaimo in \cite{DGEGCC}. For homogeneous affine
semigroup rings the conjecture holds in codimension $2$ by a result of Peeva
and Sturmfels \cite{codim2}. Even in the simplicial setting the conjecture is
open except for the isolated singularity case by Herzog and Hibi \cite{CMHH},
if the ring is seminormal by \cite{MNSN}, and for some other cases by
\cite{HSCM, MNEGMC}.\newline

We now focus on computing the regularity in the homogeneous affine semigroup
case. Homogeneity of an affine semigroup $B$ in $\mathbb{N}^{m}$ with
$\mathrm{Hilb}(B)=\{b_{1},\ldots,b_{n}\}$ is equivalent to the existence of a
group homomorphism $\mathrm{deg}: G(B)\rightarrow\mathbb{Z}$ with
$\mathrm{deg}\, b_{i}=1$ for all $i$. We always consider the $R$-module
structure on $K[B]$ given by the homogeneous surjective $K$-algebra
homomorphism $R\twoheadrightarrow K[B], x_{i}\mapsto t^{b_{i}}$. In terms of
the decomposition the regularity can be computed as follows:

\begin{proposition}
\label{regcomp}

Let $K$ be an arbitrary field and let $B\subseteq\mathbb{N}^{m}$ be a
homogeneous affine semigroup. Fix a group homomorphism $\mathrm{deg}:
G(B)\rightarrow\mathbb{Z}$ with $\mathrm{deg}\, b=1$ for all $b\in
\mathrm{Hilb}(B)$. Moreover, let $A$ be a submonoid of $B$ with $\mathrm{Hilb}%
(A)=\{e_{1},\ldots,e_{d}\}$, $\mathrm{deg}\, e_{i}=1$ for all $i$, and
$C(A)=C(B)$. Let $K[B]\cong\bigoplus_{g\in G} I_{g} (-h_{g})$ be the output of
Algorithm~\ref{algo1} with respect to $A\subseteq B$. Then

\begin{enumerate}
\item $\mathrm{reg}K[B] = \max\left\{ \mathrm{reg}I_{g} + \mathrm{deg}%
\,h_{g}\mid g\in G\right\} $; where $\mathrm{reg} I_{g}$ denotes the
regularity of the ideal $I_{g}\subseteq K[A]$ as a canonical $K[x_{1}%
,\ldots,x_{d}]$-module.

\item $\mathrm{deg}K[B] = \#G \cdot\mathrm{deg}K[A]$.
\end{enumerate}
\end{proposition}

\begin{proof}
(1) Consider the $T=K[x_{1},\ldots,x_{d}]$-module structure on $K[B]$ which is
given by $T\twoheadrightarrow K[A]\subseteq K[B]$. Since $C(A)=C(B)$ we get by
\cite[Theorem\,13.1.6]{BSLC}
\[
a(H^{i}_{R_{+}}(K[B]))=a(H^{i}_{K[B]_{+}}(K[B]))=a(H^{i}_{T_{+}}(K[B])),
\]
where $K[B]_{+}$ is the homogeneous maximal ideal of $K[B]$. Then the
assertion follows from the fact $K[B]\cong\bigoplus_{g\in G} I_{g}
(-\mathrm{deg}\,h_{g})$ as $\mathbb{Z}$-graded $T$-modules.

(2) Follows from $\mathrm{deg}I_{g}=\mathrm{deg}K[A]$ for all $g\in G$.
\end{proof}

Using Proposition~\ref{regcomp} we obtain Algorithm~\ref{algo2}.

\begin{algorithm}[H]
\caption{The regularity algorithm}
\label{algo2}
\begin{algorithmic}[1]
\REQUIRE The Hilbert basis ${\rm Hilb}(B)$ of a homogeneous affine semigroup $B\subseteq \mathbb N^m$ and a field $K$.
\ENSURE The Castelnuovo-Mumford regularity ${\rm reg}K[B]$.
\STATE Choose a minimal subset $\{e_1,\ldots,e_d\}$ of ${\rm Hilb}(B)$ with $C(\{e_1,\ldots,e_d\})=C(B)$, and set $A=\erz{e_1,\ldots,e_d}$.
\STATE Compute the decomposition $K[B]\cong\bigoplus_{g\in G} I_g(-h_g)$ over $K[A]$.
\STATE Compute a hyperplane $H=\{(t_1,\ldots,t_m)\in\mathbb R^m \mid \sum_{j=1}^m a_jt_j =c\}$ with $c\not=0$ such that ${\rm Hilb}(B)\subseteq H$. Define ${\rm deg}:\mathbb R^m\rightarrow\mathbb R$ by ${\rm deg} (t_1,\ldots,t_m) = (\sum_{j=1}^m a_jt_j)/c$.
\RETURN ${\rm reg}K[B] = \max\maxk{{\rm reg}I_g + {\rm deg}\,h_g\mid g\in G}$.
\end{algorithmic}
\end{algorithm}


By Algorithm~\ref{algo2} the computation of $\mathrm{reg}K[B]$ reduces to
computing minimal graded free resolutions of the monomial ideals $I_{g}$ in
$K[A]$ as a $K[x_{1},\ldots,x_{d}]$-module.

\begin{example}
We apply Algorithm \ref{algo2} using the decomposition computed in
Example~\ref{zerlbeispiel}. A resolution of $I=\left\langle xy,w^{2}%
z^{2}\right\rangle $ as a $T=K\left[  w,x,y,z\right]  $-module is%
\[
0\longrightarrow T\left(  -5\right)  \oplus T\left(  -6\right)  \overset
{d}{\longrightarrow}T\left(  -2\right)  \oplus T\left(  -4\right)
\longrightarrow I\longrightarrow0
\]
with%
\[
d=\left(
\begin{array}
[c]{cc}%
xy^{2} & z^{2}w^{2}\\
-w & -xy
\end{array}
\right) ,
\]
hence $\mathrm{reg}I=5$. The group homomorphism is given by $\mathrm{deg}%
\,b=(\sum_{j=1}^{4}b_{j})/5$ and therefore $\mathrm{reg}K[B]=\max\left\{
5-1,5-2\right\} =4$.
\end{example}

In case of monomial curves the ideals $I_{g}$ are monomial ideals in two
variables. Hence we can read off $\mathrm{reg}I_{g}$ by ordering the monomials
with respect to the lexicographic order (see for example
\cite[Proposition\,4.1]{MNEGMC}). This further improves the performance of the algorithm. 

With respect to timings, we focus on dimension $3$ comparing our
implementation of Algorithm~\ref{algo2} in the \textsc{Macaulay2} package \textsc{MonomialAlgebras} (marked in the tables by MA) with other methods. 
Here we consider the computation of the
regularity via a minimal graded free resolution both in \textsc{Macaulay2}
(M2) and \textsc{Singular} \cite{DGPS} (S). Furthermore, we compare with the
implementation of Bermejo, Gimenez, and Greuel in the \textsc{Singular}
package \textsc{mregular.lib} \cite{GLP2} (MREG) which does not require the
computation of a free resolution. We give the average computation times over
$n$ examples generated by the function \texttt{randomSemigroups($\alpha
$,d,c,n)}. Starting with the standard random seed, this function generates $n$
random semigroups in $\mathbb{N}^{d}$ with full dimension, coordinate sum
$\alpha$, and codimension $c$. All timings are in seconds on a single $2$ GHz
core and $4$ GB of Ram. In the cases marked by a star at least one of the
computations ran out of memory or did not finish within $1200$ seconds. Note
that the computation of $\mathrm{reg}I_{g}$ in step $4$ of
Algorithm~\ref{algo2} can be done fully parallel. This is not available in our
\textsc{Macaulay2} implementation so far.

The next table shows the comparison for $d=3$, $\alpha=5$, and $n=15$
examples. For the coefficient field we will always choose $K=\mathbb{Q}$ as
\textsc{mregular.lib} does not perform well over finite fields.
\[%
\begin{tabular}
[c]{lccccccccc}%
$c$ & \multicolumn{1}{|c}{$1$} & $2$ & $3$ & $4$ & $5$ & $6$ & $7$ & $8$ &
$9$\\\hline
MA & \multicolumn{1}{|c}{$.09$} & $.12$ & $.13$ & $.15$ & $.17$ & $.18$ &
$.19$ & $.27$ & $.25$\\
M2 & \multicolumn{1}{|c}{$.02$} & $.02$ & $.03$ & $.04$ & $.07$ & $.14$ &
$.53$ & $3.4$ & $23$\\
S & \multicolumn{1}{|c}{$.01$} & $.01$ & $.01$ & $.01$ & $.03$ & $.07$ &
$.23$ & $.95$ & $6.0$\\
MREG & \multicolumn{1}{|c}{$.01$} & $.11$ & $.57$ & $4.0$ & $32$ & $41$ &
$130$ & $\ast$ & $\ast$\\
&  &  &  &  &  &  &  &  & \\
$c$ & \multicolumn{1}{|c}{$10$} & $11$ & $12$ & $13$ & $14$ & $15$ & $16$ &
$17$ & $18$\\\hline
MA & \multicolumn{1}{|c}{$.34$} & $.47$ & $.42$ & $.44$ & $.54$ & $.53$ &
$.67$ & $.75$ & $.86$\\
M2 & \multicolumn{1}{|c}{$177$} & $\ast$ & $\ast$ & $\ast$ & $\ast$ & $\ast$
& $\ast$ & $\ast$ & $\ast$\\
S & \multicolumn{1}{|c}{$26$} & $\ast$ & $\ast$ & $\ast$ & $\ast$ & $\ast$ &
$\ast$ & $\ast$ & $\ast$\\
MREG & \multicolumn{1}{|c}{$\ast$} & $\ast$ & $\ast$ & $\ast$ & $\ast$ &
$\ast$ & $\ast$ & $\ast$ & $\ast$%
\end{tabular}
\]
For small codimension $c$ the decomposition approach has slightly higher
overhead than the traditional algorithms. For larger codimensions, however,
both the resolution approach and the Bermejo-Gimenez-Greuel implementation
fail, whereas the average computation time for Algorithm~\ref{algo2} stays
under one second.

To illustrate the performance of Algorithm~\ref{algo2} we present the
computation times ($K=\mathbb{Q}$, $n=1$) of our implementation for $d=3$ and
various $\alpha$ and $c$:%
\[%
\begin{tabular}
[c]{c|ccccccccccccc}%
$\alpha\backslash c$ & $4$ & $8$ & $12$ & $16$ & $20$ & $24$ & $28$ & $32$ &
$36$ & $40$ & $44$ & $48$ & $52$\\\hline
$3$ & $.11$ &  &  &  &  &  &  &  &  &  &  &  & \\
$4$ & $.10$ & $.14$ & $.41$ &  &  &  &  &  &  &  &  &  & \\
$5$ & $.15$ & $.37$ & $.24$ & $.44$ &  &  &  &  &  &  &  &  & \\
$6$ & $.15$ & $.78$ & $.43$ & $.66$ & $.80$ & $2.0$ &  &  &  &  &  &  & \\
$7$ & $.18$ & $.45$ & $.58$ & $1.2$ & $1.7$ & $3.2$ & $3.6$ & $6.0$ &  &  &  &
& \\
$8$ & $.14$ & $.60$ & $1.1$ & $1.6$ & $2.0$ & $3.7$ & $4.1$ & $6.4$ & $11$ &
$22$ &  &  & \\
$9$ & $.24$ & $.90$ & $1.3$ & $4.7$ & $4.1$ & $4.4$ & $13$ & $15$ & $16$ &
$32$ & $39$ & $55$ & $81$%
\end{tabular}
\ \ \
\]
The following table is based on a similar setup for $d=4$:%
\[%
\begin{tabular}
[c]{c|ccccccccccccc}%
$\alpha\backslash c$ & $4$ & $8$ & $12$ & $16$ & $20$ & $24$ & $28$ & $32$ &
$36$ & $40$ & $44$ & $48$ & $52$\\\hline
$2$ & $.16$ &  &  &  &  &  &  &  &  &  &  &  & \\
$3$ & $.15$ & $.26$ & $.29$ & $.87$ &  &  &  &  &  &  &  &  & \\
$4$ & $.15$ & $.72$ & $1.1$ & $1.1$ & $1.3$ & $2.1$ & $4.1$ &  &  &  &  &  &
\\
$5$ & $.24$ & $.85$ & $13$ & $44$ & $6.7$ & $14$ & $20$ & $16$ & $17$ & $33$ &
$42$ & $58$ & $81$\\
$6$ & $.48$ & $20$ & $220$ & $510$ & $\ast$ & $\ast$ & $\ast$ & $\ast$ &
$\ast$ & $\ast$ & $\ast$ & $\ast$ & $\ast$%
\end{tabular}
\ \ \
\]


Due to the good performance of Algorithm~\ref{algo2} we can actually do the
regularity computation for all possible semigroups $B$ in $\mathbb{N}^{d}$
such that the generators have coordinate sum $\alpha$ for some $\alpha$ and
$d$. This confirms the Eisenbud-Goto conjecture for some cases.

\begin{proposition}
\label{testegbis5} The regularity of $\mathbb{Q}[B]$ is bounded by
$\mathrm{deg}\mathbb{Q}[B]-\mathrm{codim}\mathbb{Q}[B]$, provided that the
minimal generators of $B$ in $\mathbb{N}^{d}$ have fixed coordinate sum
$\alpha$ for $d=3$ and $\alpha\leq5$, for $d=4$ and $\alpha\leq3$, as well as
for $d=5$ and $\alpha=2$.
\end{proposition}

\begin{proof}
The list of all generating sets $\mathrm{Hilb}(B)$ together with
$\mathrm{reg}\mathbb{Q}[B]$, $\mathrm{deg}\mathbb{Q}[B]$, and $\mathrm{codim}%
\mathbb{Q}[B]$ can be found under the link given in \cite{BEN}.
\end{proof}




\begin{thebibliography}{00}                                                                                                

\bibitem{BSCDM}
D.~Bayer and M. Stillman, \emph{A criterion for detecting m-regularity},
  Invent. Math. \textbf{87} (1987), no.~1, 1--11.
  
  \bibitem{GLP2} 
I. Bermejo, P. Gimenez, G.-M. Greuel,
\emph{mregular.lib}. {A} {\sc Singular} {3-1-3} library for computing the Castelnuovo-Mumford regularity of homogeneous ideals.
  
  \bibitem{BEN}J. B\"{o}hm, D. Eisenbud, and M. J. Nitsche, \emph{MonomialAlgebras -- Decomposition of positive affine semigroup rings},
Macaulay2 package, 2011, available at
\href{http://www.math.uni-sb.de/ag/schreyer/jb/Macaulay2/MonomialAlgebras/html/}{http://www.math.uni-sb.de/ag/schreyer/jb/Macaulay2/MonomialAlgebras/html/}

\bibitem{BSLC}
M.~P. Brodmann and R.~Y. Sharp, \emph{{Local cohomology: an algebraic
  introduction with geometric applications}}, Cambridge Studies in Advanced
  Mathematics, vol.~60, Cambridge University Press, Cambridge, 1998.
  
    \bibitem{DGPS}
W. Decker, G.-M. Greuel, G. Pfister, H. Sch{\"o}nemann,
\newblock {\sc Singular} {3-1-3} --- {A} computer algebra system for polynomial computations.
\newblock {http://www.singular.uni-kl.de} (2011).

\bibitem{EG}
D.~Eisenbud and S.~Goto, \emph{{Linear free resolutions and minimal
  multiplicity}}, J. Algebra \textbf{88} (1984), no.~1, 89--133.

\bibitem{GSRBB}
P.~A. {Garc\'ia-S\'anchez} and J.~C. Rosales, \emph{{On Buchsbaum simplicial
  affine semigroups}}, Pacific J. Math. \textbf{202} (2002), no.~2, 329--339.

\bibitem{DGEGCC}
D.~Giaimo, \emph{{On the Castelnuovo-Mumford regularity of connected curves}},
  Trans. Amer. Math. Soc. \textbf{358} (2006), no.~1, 267--284.

\bibitem{M2}
D.~R. Grayson and M.~E. Stillman, \emph{Macaulay2, a software system for
  research in algebraic geometry}, Available at
  http://www.math.uiuc.edu/Macaulay2/.

\bibitem{GLP}
L.~Gruson, R.~Lazarsfeld, and C.~Peskine, \emph{{On a theorem of Castelnuovo,
  and the equations defining space curves}}, Invent. Math. \textbf{72} (1983),
  no.~3, 491--506.

\bibitem{CMHH}
J.~Herzog and T.~Hibi, \emph{{Castelnuovo-Mumford regularity of simplicial
  semigroup rings with isolated singularity}}, Proc. Amer. Math. Soc.
  \textbf{131} (2003), no.~9, 2641--2647.

\bibitem{HSCM}
L.~T. Hoa and J.~St{\"u}ckrad, \emph{{Castelnuovo-Mumford regularity of
  simplicial toric rings}}, J. Algebra \textbf{259} (2003), no.~1, 127--146.

\bibitem{HSVC2}
L.~T. Hoa, J.~St\"uckrad, and W.~Vogel, \emph{Towards a structure theory for
  projective varieties of degree = codimension + 2}, J. Pure Appl. Algebra
  \textbf{71} (1991), no.~2-3, 203--231.

\bibitem{MHCM}
M.~Hochster, \emph{{Rings of invariants of tori, Cohen-Macaulay rings generated
  by monomials, and polytopes}}, Ann. Math. \textbf{96} (1972), no.~2,
  318--337.

\bibitem{HRLC}
M.~Hochster and J.~L. Roberts, \emph{{The purity of the Frobenius and local
  cohomology}}, Adv. Math. \textbf{21} (1976), no.~2, 117--172.

\bibitem{LSMSD3}
R.~Lazarsfeld, \emph{{A sharp Castelnuovo bound for smooth surfaces}}, Duke
  Math. J. \textbf{55} (1987), no.~2, 423--429.

\bibitem{PHDPL}
P.~Li, \emph{{Seminormality and the Cohen-Macaulay property}}, Ph.D. thesis,
  Queen's University, Kingston, Canada, 2004.

\bibitem{MNSN}
M.~J. Nitsche, \emph{{Castelnuovo-Mumford regularity of seminormal simplicial
  affine semigroup rings}}, MPI MIS Preprint 61/2010, 2010.

\bibitem{MNEGMC}
M.~J. Nitsche, \emph{{A combinatorial proof of the Eisenbud-Goto conjecture for
  monomial curves and some simplicial semigroup rings}}, MPI MIS Preprint
  10/2011, 2011.

\bibitem{codim2}
I.~Peeva and B.~Sturmfels, \emph{Syzygies of codimension 2 lattice ideals},
  Math. Z. \textbf{229} (1998), no.~1, 163--194.
  
\bibitem{RSHFGA}
R.~P. Stanley, \emph{Hilbert functions of graded algebras}, Adv. Math.
  \textbf{28} (1978), no.~1, 57--83.

\bibitem{EGBB}
J.~St\"uckrad and W.~Vogel, \emph{{Castelnuovo bounds for certain subvarieties
  in $\mathbb P^n$}}, Math. Ann. \textbf{276} (1987), no.~2, 341--352.

\bibitem{EGCM}
R.~Treger, \emph{{On equations defining arithmetically Cohen-Macaulay schemes.
  I}}, Math. Ann. \textbf{261} (1982), no.~2, 141--153.


\end{thebibliography}


\end{document}