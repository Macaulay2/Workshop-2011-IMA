\documentclass{amsart}
\usepackage{enumerate}
\usepackage{amssymb}
\usepackage{graphicx}
%\usepackage{draftcopy}

\usepackage[paper=letterpaper,width=7in,height=9.25in,centering]{geometry}

\title{Working notes: Primary decomposition algorithms}
\author{Franziska Hinkelmann, Frank Moore, and Mike Stillman}
\date{\today}

\theoremstyle{definition}
\newtheorem{theorem}{Theorem}
\newtheorem{lemma}[theorem]{Lemma}
\newtheorem{definition}[theorem]{Definition}
\newtheorem{proposition}[theorem]{Proposition}
\newtheorem{corollary}[theorem]{Corollary}
\newtheorem{example}[theorem]{Example}
\newtheorem{algorithm}[theorem]{Algorithm}

\DeclareMathOperator{\sigSym}{sig}
\DeclareMathOperator{\lcm}{lcm}
\DeclareMathOperator{\tr}{tr}
\DeclareMathOperator{\Hom}{Hom}
\DeclareMathOperator{\Gal}{Gal}
\newcommand{\mmorSym}{\phi}

\newcommand{\R}{\mathbb{R}}
\newcommand{\Q}{\mathbb{Q}}
\newcommand{\Z}{\mathbb{Z}}
\newcommand{\N}{\mathbb{N}}
\newcommand{\q}{\mathfrak{q}}
\newcommand{\ra}{\rightarrow}
\newcommand{\La}{{\mathcal{L}_p}}
\newcommand{\s}{\ensuremath{^*}}
\newcommand{\p}{\ensuremath{^\prime}}
\newcommand{\pp}{\ensuremath{^{\prime\prime}}}
\newcommand{\biimp}{\Leftrightarrow}
\newcommand{\imp}{\Rightarrow}
\newcommand{\impby}{\Leftarrow}
\newcommand{\defeq}{\stackrel{\text{\tiny def}}{=}}
\newcommand{\proj}[1]{\overline{#1}}
\newcommand{\mbasis}[1]{\boldsymbol{e}_{#1}}
\newcommand{\sig}[1]{\sigSym\left({#1}\right)}
\newcommand{\origBasis}{\mathcal G}
\newcommand{\extendBasis}[1]{\origBasis_{#1}}
\newcommand{\mmor}[1]{\mmorSym\left({#1}\right)}
\newcommand{\Rone}{(\text{R}_1)}
\newcommand{\Stwo}{(\text{S}_2)}

\newcommand{\set}[1]{\left\{{#1}\right\}}
\newcommand{\setBuilder}[2]{\left\{{#1}\left|{#2}\right.\right\}}
\newcommand{\setBuilderBarLeft}[2]{\left\{\left.{#1}\right|{#2}\right\}}
\newcommand{\idealBuilder}[2]{\left\langle#1\left|#2\right.\right\rangle}
\newcommand{\idealBuilderBarLeft}[2]{\left\langle\left.#1\right|#2\right\rangle}
\newcommand{\ideal}[1]{\left\langle{#1}\right\rangle}
\newcommand{\hd}[1]{\text{in}(#1)}
\newcommand{\hdp}[1]{\hd{\proj{#1}}}
\newcommand{\card}[1]{\left|#1\right|}
\newcommand{\spair}[2]{\mathcal{S}({#1},{#2})}

\newcommand{\grobner}{Gr\"obner}
\newcommand{\galg}{signature Buchberger algorithm}
\newcommand{\varProd}{\mathbbm{x}}

\newcommand{\proofPart}[1]{{\bf $\boldsymbol{#1}$:}}
\newcommand{\proofSubPart}[1]{{\underline{${#1}$:}}}
\newcommand{\proofCase}[1]{\proofPart{\text{The case }#1}}

\begin{document}

\maketitle
\setcounter{tocdepth}{1}
\tableofcontents

%notational conventions: S = Noeth ring we want to Normalize
%                        R = Noether normalization of S
%                        A = arbitrary Noetherian ring (i.e. used in proofs where the ring is not thought of as S or R)

\section{Introduction}

Let $k$ be a field, and let $R = k[x_1, \ldots, x_n]$ be a polynomial ring.  
If $I \subset R$ is an ideal, there are a number of algorithms to compute the
set of minimal primes of $I$ (cite...).  In these notes, we describe in detail
our algorithms, including different strategies which we use.

\section{Splitting methods}

\subsection{Monomials}

\begin{proposition}
Suppose that $I = J + L$, where $I, J, L \subset R = k[x_1, \ldots, x_n]$ are ideals.
If the minimal primes of $J$ are $P_1, \ldots, P_N$, then 
  $$\sqrt{I} = \sqrt{P_1 + L} \,\cap\, \ldots \,\cap\, \sqrt{P_N + L}.$$
\end{proposition}

\bigskip

We apply this Proposition in the following manner.  If $I = (f_1,
\ldots, f_r, f_{r+1}, \ldots, f_s)$, where $f_i$ is a monomial for $1
\leq i \leq r$, let $J = (f_1, \ldots, f_r)$, and $L = (f_{r+1},
\ldots, f_s)$.  We use Alexander dual methods (see
\cite{frobby-paper}) to compute the minimal primes of the monomial
ideal $J$.  This last algorithm is extremely fast in practice.

A disadvantage of this method is that the resulting decomposition may be redundant.

\subsection{IndependentSets}



\subsection{Factorization}

\bibliographystyle{plain}
\bibliography{references}
\begin{thebibliography}{2}

\bibitem{AM}
  Atiyah-Macdonald
\bibitem{Na}
  Nagata - Local Rings
\bibitem{Tr}
  Trager's Thesis
\bibitem{Va}
  Vasconcelos - Computations
\bibitem{frobby-paper}
  Roune, The slice algorithm
\end{thebibliography}
\end{document}
